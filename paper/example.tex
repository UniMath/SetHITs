\documentclass[9pt]{entcs}

\usepackage{entcsmacro}
\usepackage{graphicx}
\usepackage{amsmath}
\usepackage{bussproofs}
\usepackage{stmaryrd}

\sloppy
% The following is enclosed to allow easy detection of differences in
% ascii coding.
% Upper-case    A B C D E F G H I J K L M N O P Q R S T U V W X Y Z
% Lower-case    a b c d e f g h i j k l m n o p q r s t u v w x y z
% Digits        0 1 2 3 4 5 6 7 8 9
% Exclamation   !           Double quote "          Hash (number) #
% Dollar        $           Percent      %          Ampersand     &
% Acute accent  '           Left paren   (          Right paren   )
% Asterisk      *           Plus         +          Comma         ,
% Minus         -           Point        .          Solidus       /
% Colon         :           Semicolon    ;          Less than     <
% Equals        =3D           Greater than >          Question mark ?
% At            @           Left bracket [          Backslash     \
% Right bracket ]           Circumflex   ^          Underscore    _
% Grave accent  `           Left brace   {          Vertical bar  |
% Right brace   }           Tilde        ~


\newenvironment{bprooftree}
{\leavevmode\hbox\bgroup}
{\DisplayProof\egroup}

\newcommand{\type}[1]{#1}
\newcommand{\constructor}[1]{#1}
\newcommand{\category}[1]{#1}
\newcommand{\functortxt}[1]{#1}
\newcommand{\nattranstxt}[1]{#1}

\newcommand{\hset}{\type{hSet}} % sets
\newcommand{\1}{\type{1}} % sets

% Category theory
\newcommand{\setoids}{\category{Setoid}}
\newcommand{\functor}[2]{#1 \rightarrow #2} % functors
\newcommand{\idf}{\functortxt{id}} % identity functor
\newcommand{\compf}[2]{#1 \circ #2} % composition of functors
\newcommand{\prodf}[2]{#1 \times #2} % product of functors
\newcommand{\sumf}[2]{#1 + #2} % sum of functors
\newcommand{\nattrans}[2]{#1 \Rightarrow #2} % natural transfomations
\newcommand{\idt}[1]{\nattranstxt{id}} % identity transformation
\newcommand{\compt}[2]{#1 \circ #2} % composition of natural transformations
\newcommand{\lwhisker}[2]{#1 \vartriangleleft #2} % left whiskering of transformations
\newcommand{\inlt}{\nattranstxt{inl}} %left inclusion of transformations
\newcommand{\inrt}{\nattranstxt{inr}} % right inclusion of transformations
\newcommand{\prlt}{\nattranstxt{pr}_1} % first projection of transformations
\newcommand{\prrt}{\nattranstxt{pr}_2} % second projection of transformations
\newcommand{\pairt}[2]{(#1 , #2)} % pairing of transformations

% Polynomials
\newcommand{\poly}{\mathcal{P}} % type of polynomials
\newcommand{\C}{\constructor{C}} % constant polynomial
\newcommand{\I}{\constructor{I}} % identity polynomial
\newcommand{\sumP}[2]{#1 + #2} % sum of polynomials
\newcommand{\prodP}[2]{#1 \times #2} % product of polynomials

% Endpoints
\newcommand{\ep}[3]{\mathcal{E}_{#1}(#2,#3)} % type of endpoinst
\newcommand{\id}[1]{\constructor{id}_{#1}} % identity endpoints
\newcommand{\comp}[2]{#1 \cdot #2} % composition of endpoints
\newcommand{\inle}{\constructor{inl}} %left inclusion of endpoints
\newcommand{\inre}{\constructor{inr}} % right inclusion of endpoints
\newcommand{\prle}{\constructor{pr}_1} % first projection of endpoints
\newcommand{\prre}{\constructor{pr}_2} % second projection of endpoints
\newcommand{\pair}[2]{(#1 , #2)} % pairing of endpoints
\newcommand{\Ce}{\constructor{c}} % constant endpoint
\newcommand{\constr}{\constructor{constr}} % constructor endpoint

% Algebras
\newcommand{\semP}[1]{\llbracket #1 \rrbracket} % semantics of polynomials
\newcommand{\prealg}[1]{\category{PreAlg}(#1)} % category of prealgebras
\newcommand{\forget}[1]{F_{#1}} % forgetful functor
\newcommand{\semE}[1]{\llbracket #1 \rrbracket} % semantics of endpoints

% Algebras in setoids
\newcommand{\semPT}[1]{\langle #1 \rangle} % semantics of polynomials in setoids
\newcommand{\semET}[1]{\langle #1 \rangle} % semantics of endpoints in setoids

\def\lastname{Please list Your Lastname Here}
\begin{document}
\begin{frontmatter}
  \title{An Example Paper} \author{My
    Name\thanksref{ALL}\thanksref{myemail}}
  \address{My Department\\ My University\\
    My City, My Country} \author{My Co-author\thanksref{coemail}}
  \address{My Co-author's Department\\My Co-author's University\\
    My Co-author's City, My Co-author's Country} \thanks[ALL]{Thanks
    to everyone who should be thanked} \thanks[myemail]{Email:
    \href{mailto:myuserid@mydept.myinst.myedu} {\texttt{\normalshape
        myuserid@mydept.myinst.myedu}}} \thanks[coemail]{Email:
    \href{mailto:couserid@codept.coinst.coedu} {\texttt{\normalshape
        couserid@codept.coinst.coedu}}}
\begin{abstract} 
  This is a short example to show the basics of using the ENTCS style
  macro files.  Ample examples of how files should look may be found
  among the published volumes of the series at the ENTCS Home Page
  \texttt{http://www.elsevier.com/locate/entcs}.
\end{abstract}
\begin{keyword}
  Please list keywords from your paper here, separated by commas.
\end{keyword}
\end{frontmatter}
\section{Introduction}
\label{intro}

\subsection*{Preliminaries}

\section{Signature of HITs}

\begin{definition}
We define the type $\mathcal{\poly}$ inductively by the following rules
\begin{center}
\begin{bprooftree}
\AxiomC{$X : \hset$}
\UnaryInfC{$\C \> X : \poly$}
\end{bprooftree}
\begin{bprooftree}
\AxiomC{$\I : \poly$}
\end{bprooftree}
\begin{bprooftree}
\AxiomC{$P : \poly$}
\AxiomC{$Q : \poly$}
\BinaryInfC{$\sumP{P}{Q} : \poly$}
\end{bprooftree}
\begin{bprooftree}
\AxiomC{$P : \poly$}
\AxiomC{$Q : \poly$}
\BinaryInfC{$\prodP{P}{Q} : \poly$}
\end{bprooftree}
\end{center}
\end{definition}

\begin{definition}
Given polynomials, $A, P, Q : \poly$, we define the type $\ep{A}{P}{Q}$ of endpoints inductively by the following rules
\begin{center}
\begin{bprooftree}
\AxiomC{$P : \poly$}
\UnaryInfC{$\id{A} : \ep{A}{P}{P}$}
\end{bprooftree}
\begin{bprooftree}
\AxiomC{$P, Q, R : \poly$}
\AxiomC{$e_1 : \ep{A}{P}{Q}$}
\AxiomC{$e_2 : \ep{A}{Q}{R}$}
\TrinaryInfC{$\comp{e_1}{e_2} : \ep{A}{P}{R}$}
\end{bprooftree}
\end{center}
\begin{center}
\begin{bprooftree}
\AxiomC{$P, Q : \poly$}
\UnaryInfC{$\inle : \ep{A}{P}{\sumP{P}{Q}}$}
\end{bprooftree}
\begin{bprooftree}
\AxiomC{$P, Q : \poly$}
\UnaryInfC{$\inre : \ep{A}{Q}{\sumP{P}{Q}}$}
\end{bprooftree}
\begin{bprooftree}
\AxiomC{$P, Q : \poly$}
\UnaryInfC{$\prle : \ep{A}{\prodP{P}{Q}}{P}$}
\end{bprooftree}
\begin{bprooftree}
\AxiomC{$P, Q : \poly$}
\UnaryInfC{$\prre : \ep{A}{\prodP{P}{Q}}{Q}$}
\end{bprooftree}
\end{center}
\begin{center}
\begin{bprooftree}
\AxiomC{$\constr : \ep{A}{A}{\I}$}
\end{bprooftree}
\begin{bprooftree}
\AxiomC{$P : \poly$}
\AxiomC{$X : \hset$}
\AxiomC{$x : X$}
\TrinaryInfC{$\Ce \> x : \ep{A}{P}{\C \> X}$}
\end{bprooftree}
\begin{bprooftree}
\AxiomC{$P, Q, R: \poly$}
\AxiomC{$e_1 : \ep{A}{P}{Q}$}
\AxiomC{$e_2 : \ep{A}{P}{R}$}
\TrinaryInfC{$\pair{e_1}{e_2} : \ep{A}{P}{\prodP{Q}{R}}$}
\end{bprooftree}
\end{center}
\end{definition}

\begin{definition}
A signature of a higher inductive type consists of
\begin{enumerate}
	\item A polynomial $P : \poly$;
	\item A type $J$;
	\item A family $Q : J \rightarrow \poly$;
	\item For each $j : J$, endpoints $l, r : \ep{P}{Q \> j}{I}$.
\end{enumerate}
\end{definition}

\begin{example}
Set with assocative operation.
\end{example}

Similarly, one can define signatures for groups or finite sets.
Note that signatures can also depend on types.

\begin{example}
Propositional truncation of a type
\end{example}

\section{Algebras}
Note: algebra structure is point constructor rule.

\begin{definition}
For each $P : \poly$, we define a functor $\semP{P} : \functor{\hset}{\hset}$. On the objects, they act as follows
\begin{itemize}
	\item $\semP{\C \> X} \> Z = X$;
	\item $\semP{\I} \> Z = Z$;
	\item $\semP{\sumP{P}{Q}} \> Z = \semP{P} \> Z + \semP{Q} \> Z$;
	\item $\semP{\sumP{P}{Q}} \> Z = \semP{P} \> Z \times \semP{Q} \> Z$.
\end{itemize}
\end{definition}

\begin{definition}
For each $P : \poly$, we define $\prealg{P}$ as the category on algebras on $\semP{P}$.
\end{definition}

Note that we have a functor $\forget{P} : \functor{\semP{P}}{\hset}$.

To define the interpretation of endpoints, we first note that for every two functors $F, G$ we have a natural transformation $\inlt : \nattrans{F}{\sumf{F}{G}}$, and we also have operations representing the other inclusion and the projections.
In addition, given $\eta_1 : \nattrans{\compf{F}{G_1}}{\compf{F}{G_2}}$ and $\eta_2 : : \nattrans{\compf{F}{G_1}}{\compf{F}{G_3}}$, we get $\pairt{\eta_1}{\eta_2} : \nattrans{\compf{F}{G_1}}{\compf{F}{\prodf{G_2}{G_3}}}$.

\begin{definition}
For each endpoint $e : \ep{A}{P}{Q}$, we define a natural transformation $\semE{e} : \nattrans{\compf{\forget{A}}{\semP{P}}} {\compf{\forget{A}}{\semP{Q}}}$
\begin{itemize}
	\item $\semE{\id{P}} = \idt{\compf{\forget{A}}{\semP{P}}}$;
	\item $\semE{\comp{e_1}{e_2}} = \compt{\semE{e_1}}{\semE{e_2}}$;
	\item $\semE{\inle} = \lwhisker{\forget{A}}{\inlt}$;
	\item $\semE{\inre} = \lwhisker{\forget{A}}{\inrt}$;
	\item $\semE{\prle} = \lwhisker{\forget{A}}{\prlt}$;
	\item $\semE{\prre} = \lwhisker{\forget{A}}{\prrt}$;
	\item $\semE{\pair{e_1}{e_2}} = \pairt{\semE{e_1}}{\semE{e_2}}$;
	\item $\semE{\Ce \> t} = $;
	\item $\semE{\constr} = $.
\end{itemize}
\end{definition}

\begin{definition}
Let $\Sigma$ be a HIT signature.
Then we define the category of \emph{$\Sigma$-algebras} as the full subcategory of $\prealg{}$ with respect to
\end{definition}

Since the category of prealgebra is univalent and univalence is preserved under taking subcategories, we conclude that this category is univalent.

\begin{proposition}
The category of algebras is univalent.
\end{proposition}

Next we define algebras in setoids.
For these, we take a slightly different approach.
Instead of defining the interpretation of the polynomial $P$ inductively making use of binary sums and products in $\setoids$, we define an equivalence relation on $\semP{P}$, and show this action is functorial.

\begin{definition}
Let $R$ be an equivalence relation on a set $X$ and let $P : \poly$ be a polynomial.
By induction, we define an equivalence relation $P \> R$ on $\semP{P} \> X$.
\begin{itemize}
	\item
	\item
	\item
	\item
\end{itemize}
Now we define a functor $\semPT{P} : \functor{\setoids}{\setoids}$ for which $\semPT{P}(X,R) = (\semP{P} \> X, P \> R)$.
\end{definition}

Again this gives rise to a category $\prealg{P}$

\begin{definition}
Let $e : \ep{A}{P}{Q}$ be an endpoint and let $X$ be a setoid prealgebra on $A$.
Then we define $\semE{e}$ to be the setoid morphism from $\semE{P}(X)$ to $\semE{Q}(X)$ whose carrier is $\semP{e}$.
\end{definition}

\begin{definition}
Let $\Sigma$ be a HIT signature.
Then we define the category of \emph{$\Sigma$-setoid-algebras} as the full subcategory of $\prealg{}$ with respect to
\end{definition}

\begin{proposition}
The category of $\Sigma$-setoid-algebras is unvialent.
\end{proposition}

\section{Initial Algebra Semantics}

\subsection{The Induction Principle}

\begin{definition}
Action of polynomials on families.
\end{definition}

\begin{definition}
Dependent action of endpoints
\end{definition}

\begin{definition}
Displayed algebra
\end{definition}

\begin{definition}
Action of polynomials on dependent maps
\end{definition}

\begin{definition}
Section for displayed algebra
\end{definition}

\begin{definition}
A HIT: every displayed algebra has a section
\end{definition}

\subsection{Obtaining Induction from Initiality}
Now suppose that we have some displayed algebra.
Our goal is to construct an algebra whose objects are dependent pairs.

Give the carrier and operation in plain text

\begin{lemma}
pr1 endpoint
\end{lemma}

\begin{lemma}
pr2 endpoint
\end{lemma}

All in all, we get the following construction.

\begin{theorem}
Total algebra and projection
\end{theorem}

\begin{proposition}
Map to total algebra which projects into identity is section.
\end{proposition}

Hence, to construct a HIT, it suffices to construct an initial object in the category of algebras.

\begin{corollary}
	
\end{corollary}

\section{Constructing the Initial Algebra}

\subsection{Adjunction between Algebras}
To construct the adjunction between the category of algebras, we use two lemmata

\begin{lemma}
Adjunction between prealgebras
\end{lemma}

\begin{lemma}
Adjunction between full subcategories
\end{lemma}

All in all, we get

\begin{proposition}
We have an adjunction
\end{proposition}

\subsection{Initial Setoid Algebra}

Hence, we conclude

\begin{theorem}
HITs exist
\end{theorem}

\section{Consequences}

\subsection{Recursion}

\subsection{Uniqueness of HITs}

\begin{proposition}
Induction implies initiality
\end{proposition}

\begin{corollary}
HITs are initial objects
\end{corollary}

\subsection{Path Spaces of HITs}

\section{Conclusion}

\begin{thebibliography}{10}\label{bibliography}
\bibitem{cy} Civin, P., and B. Yood, \emph{Involutions on Banach
    algebras}, Pacific J. Math. \textbf{9} (1959), 415--436.
  
\bibitem{cp} Clifford, A. H., and G. B. Preston, ``The Algebraic
  Theory of Semigroups,'' Math. Surveys \textbf{7}, Amer. Math. Soc.,
  Providence, R.I., 1961.
  
\bibitem{f} Freyd, Peter, Peter O'Hearn, John Power, Robert Tennent
  and Makoto Takeyama, \emph{Bireflectivity}, Electronic Notes in
  Theoretical Computer Science {\bf 1} (1995), URL:
  \href{https://www.sciencedirect.com/journal/electronic-notes-in-theoretical-computer-science/vol/1/suppl/C}
  {\texttt{http://www.elsevier.com/locate/entcs/volume1.html}}.
  
\bibitem{em2} Easdown, D., and W. D. Munn, \emph{Trace functions on
    inverse semigroup algebras}, U. of Glasgow, Dept. of Math.,
  preprint 93/52.

\bibitem{r} Roscoe, A. W., ``The Theory and Practice of Concurrency,''
  Prentice Hall Series in Computer Science, Prentice Hall Publishers,
  London, New York (1198), 565pp. With associated web site\\  
  \href{http://www.comlab.ox.ac.uk/oucl/publications/books/concurrency/}
  {\texttt{http://www.comlab.ox.ac.uk/oucl/publications/books/concurrency/}}.
  
\bibitem{s} Shehadah, A. A., ``Embedding theorems for semigroups with
  involution, `` Ph.D.  thesis, Purdue University, Indiana, 1982.
  
\bibitem{w} Weyl, H., ``The Classical Groups,'' 2nd Ed., Princeton U.
  Press, Princeton, N.J., 1946.

\end{thebibliography}

\end{document}
