\documentclass[9pt]{entcs} \usepackage{entcsmacro}
\usepackage{graphicx}
\sloppy
% The following is enclosed to allow easy detection of differences in
% ascii coding.
% Upper-case    A B C D E F G H I J K L M N O P Q R S T U V W X Y Z
% Lower-case    a b c d e f g h i j k l m n o p q r s t u v w x y z
% Digits        0 1 2 3 4 5 6 7 8 9
% Exclamation   !           Double quote "          Hash (number) #
% Dollar        $           Percent      %          Ampersand     &
% Acute accent  '           Left paren   (          Right paren   )
% Asterisk      *           Plus         +          Comma         ,
% Minus         -           Point        .          Solidus       /
% Colon         :           Semicolon    ;          Less than     <
% Equals        =3D           Greater than >          Question mark ?
% At            @           Left bracket [          Backslash     \
% Right bracket ]           Circumflex   ^          Underscore    _
% Grave accent  `           Left brace   {          Vertical bar  |
% Right brace   }           Tilde        ~

% A couple of exemplary definitions:

\newcommand{\Nat}{{\mathbb N}}
\newcommand{\Real}{{\mathbb R}}
\def\lastname{Please list Your Lastname Here}
\begin{document}
\begin{frontmatter}
  \title{An Example Paper} \author{My
    Name\thanksref{ALL}\thanksref{myemail}}
  \address{My Department\\ My University\\
    My City, My Country} \author{My Co-author\thanksref{coemail}}
  \address{My Co-author's Department\\My Co-author's University\\
    My Co-author's City, My Co-author's Country} \thanks[ALL]{Thanks
    to everyone who should be thanked} \thanks[myemail]{Email:
    \href{mailto:myuserid@mydept.myinst.myedu} {\texttt{\normalshape
        myuserid@mydept.myinst.myedu}}} \thanks[coemail]{Email:
    \href{mailto:couserid@codept.coinst.coedu} {\texttt{\normalshape
        couserid@codept.coinst.coedu}}}
\begin{abstract} 
  This is a short example to show the basics of using the ENTCS style
  macro files.  Ample examples of how files should look may be found
  among the published volumes of the series at the ENTCS Home Page
  \texttt{http://www.elsevier.com/locate/entcs}.
\end{abstract}
\begin{keyword}
  Please list keywords from your paper here, separated by commas.
\end{keyword}
\end{frontmatter}
\section{Introduction}
\label{intro}

\subsection*{Preliminaries}

\section{Signature of HITs}

\begin{definition}
Code for polynomials
\end{definition}

\begin{definition}
Endpoints
\end{definition}

\begin{definition}
Signature of a HIT
\end{definition}

\section{Algebras}
Note: algebra structure is point constructor rule.

\begin{definition}
Semantics of a polynomial on sets
\end{definition}

\begin{definition}
Category of prealgebras on a polynomial
\end{definition}

\begin{definition}
Semantics of endpoints
\end{definition}

\begin{definition}
Category of algebras
\end{definition}

\begin{proposition}
The category of algebras is univalent.
\end{proposition}

\begin{definition}
Semantics of a polynomial on setoids
\end{definition}

\begin{definition}
Category of prealgebras on a polynomial (setoids)
\end{definition}

\begin{definition}
Semantics of endpoints (setoids)
\end{definition}

\begin{definition}
Category of algebras in setoids
\end{definition}

\section{Initial Algebra Semantics}

\subsection{The Induction Principle}

\begin{definition}
Displayed algebra
\end{definition}

\begin{definition}
Section for displayed algebra
\end{definition}

\begin{definition}
A HIT: every displayed algebra has a section
\end{definition}

\subsection{Obtaining Induction from Initiality}

\begin{definition}
Total algebra
\end{definition}

\begin{definition}
Projection of total algebra
\end{definition}

\begin{proposition}
Map to total algebra which projects into identity is section.
\end{proposition}

Hence, to construct a HIT, it suffices to construct an initial object in the category of algebras.

\section{Constructing the Initial Algebra}

\subsection{Adjunction between Algebras}
To construct the adjunction between the category of algebras, we use two lemmata

\begin{lemma}
Adjunction between algebras
\end{lemma}

\begin{lemma}
Adjunction between full subcategories
\end{lemma}

\subsection{Initial Setoid Algebra}

Hence, we conclude

\begin{theorem}
HITs exist
\end{theorem}

\section{Consequences}

\subsection{Recursion}

\subsection{Uniqueness of HITs}

\begin{proposition}
Induction implies initiality
\end{proposition}

\begin{corollary}
HITs are initial objects
\end{corollary}

\subsection{Path Spaces of HITs}

\section{Conclusion}

\begin{thebibliography}{10}\label{bibliography}
\bibitem{cy} Civin, P., and B. Yood, \emph{Involutions on Banach
    algebras}, Pacific J. Math. \textbf{9} (1959), 415--436.
  
\bibitem{cp} Clifford, A. H., and G. B. Preston, ``The Algebraic
  Theory of Semigroups,'' Math. Surveys \textbf{7}, Amer. Math. Soc.,
  Providence, R.I., 1961.
  
\bibitem{f} Freyd, Peter, Peter O'Hearn, John Power, Robert Tennent
  and Makoto Takeyama, \emph{Bireflectivity}, Electronic Notes in
  Theoretical Computer Science {\bf 1} (1995), URL:
  \href{https://www.sciencedirect.com/journal/electronic-notes-in-theoretical-computer-science/vol/1/suppl/C}
  {\texttt{http://www.elsevier.com/locate/entcs/volume1.html}}.
  
\bibitem{em2} Easdown, D., and W. D. Munn, \emph{Trace functions on
    inverse semigroup algebras}, U. of Glasgow, Dept. of Math.,
  preprint 93/52.

\bibitem{r} Roscoe, A. W., ``The Theory and Practice of Concurrency,''
  Prentice Hall Series in Computer Science, Prentice Hall Publishers,
  London, New York (1198), 565pp. With associated web site\\  
  \href{http://www.comlab.ox.ac.uk/oucl/publications/books/concurrency/}
  {\texttt{http://www.comlab.ox.ac.uk/oucl/publications/books/concurrency/}}.
  
\bibitem{s} Shehadah, A. A., ``Embedding theorems for semigroups with
  involution, `` Ph.D.  thesis, Purdue University, Indiana, 1982.
  
\bibitem{w} Weyl, H., ``The Classical Groups,'' 2nd Ed., Princeton U.
  Press, Princeton, N.J., 1946.

\end{thebibliography}

\end{document}
