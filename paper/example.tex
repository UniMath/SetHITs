\documentclass[9pt]{entcs}

\usepackage{entcsmacro}
\usepackage{graphicx}
\usepackage{amsmath}
\usepackage{bussproofs}
\usepackage{stmaryrd}

\sloppy
% The following is enclosed to allow easy detection of differences in
% ascii coding.
% Upper-case    A B C D E F G H I J K L M N O P Q R S T U V W X Y Z
% Lower-case    a b c d e f g h i j k l m n o p q r s t u v w x y z
% Digits        0 1 2 3 4 5 6 7 8 9
% Exclamation   !           Double quote "          Hash (number) #
% Dollar        $           Percent      %          Ampersand     &
% Acute accent  '           Left paren   (          Right paren   )
% Asterisk      *           Plus         +          Comma         ,
% Minus         -           Point        .          Solidus       /
% Colon         :           Semicolon    ;          Less than     <
% Equals        =3D           Greater than >          Question mark ?
% At            @           Left bracket [          Backslash     \
% Right bracket ]           Circumflex   ^          Underscore    _
% Grave accent  `           Left brace   {          Vertical bar  |
% Right brace   }           Tilde        ~


\newenvironment{bprooftree}
{\leavevmode\hbox\bgroup}
{\DisplayProof\egroup}

% Setting modes
\newcommand{\type}[1]{#1}
\newcommand{\term}[1]{#1}
\newcommand{\constructor}[1]{#1}
\newcommand{\category}[1]{#1}
\newcommand{\functortxt}[1]{#1}
\newcommand{\nattranstxt}[1]{#1}
\newcommand{\function}[1]{#1}

% HoTT
\newcommand{\deprod}[3]{\prod_{#1 : #2} #3}
\newcommand{\hset}{\type{hSet}} % sets
\newcommand{\1}{\type{1}} % sets
\newcommand{\inl}{\term{inl}} %left inclusion
\newcommand{\inr}{\term{inr}} % right inclusion
\newcommand{\prl}{\term{pr}_1} % first projection
\newcommand{\prr}{\term{pr}_2} % second projection
\newcommand{\pairTm}[2]{(#1 , #2)} % pairing

% Category theory
\newcommand{\setoids}{\category{Setoid}}
\newcommand{\functor}[2]{#1 \rightarrow #2} % functors
\newcommand{\idf}{\functortxt{id}} % identity functor
\newcommand{\compf}[2]{#1 \circ #2} % composition of functors
\newcommand{\prodf}[2]{#1 \times #2} % product of functors
\newcommand{\sumf}[2]{#1 + #2} % sum of functors
\newcommand{\nattrans}[2]{#1 \Rightarrow #2} % natural transfomations
\newcommand{\idt}[1]{\nattranstxt{id}} % identity transformation
\newcommand{\compt}[2]{#1 \circ #2} % composition of natural transformations
\newcommand{\lwhisker}[2]{#1 \vartriangleleft #2} % left whiskering of transformations
\newcommand{\inlt}{\nattranstxt{inl}} %left inclusion of transformations
\newcommand{\inrt}{\nattranstxt{inr}} % right inclusion of transformations
\newcommand{\prlt}{\nattranstxt{pr}_1} % first projection of transformations
\newcommand{\prrt}{\nattranstxt{pr}_2} % second projection of transformations
\newcommand{\pairt}[2]{(#1 , #2)} % pairing of transformations
\newcommand{\idtoiso}{\function{idtoiso}}

% Polynomials
\newcommand{\poly}{\mathcal{P}} % type of polynomials
\newcommand{\C}{\constructor{C}} % constant polynomial
\newcommand{\I}{\constructor{I}} % identity polynomial
\newcommand{\sumP}[2]{#1 + #2} % sum of polynomials
\newcommand{\prodP}[2]{#1 \times #2} % product of polynomials

% Endpoints
\newcommand{\ep}[3]{\mathcal{E}_{#1}(#2,#3)} % type of endpoinst
\newcommand{\id}[1]{\constructor{id}_{#1}} % identity endpoints
\newcommand{\comp}[2]{#1 \cdot #2} % composition of endpoints
\newcommand{\inle}{\constructor{inl}} %left inclusion of endpoints
\newcommand{\inre}{\constructor{inr}} % right inclusion of endpoints
\newcommand{\prle}{\constructor{pr}_1} % first projection of endpoints
\newcommand{\prre}{\constructor{pr}_2} % second projection of endpoints
\newcommand{\pair}[2]{(#1 , #2)} % pairing of endpoints
\newcommand{\Ce}{\constructor{c}} % constant endpoint
\newcommand{\constr}{\constructor{constr}} % constructor endpoint

% Signatures
\newcommand{\pt}[0]{pt}
\newcommand{\pthI}[0]{pthI}
\newcommand{\pthA}[0]{ptha}
\newcommand{\pthlh}[0]{pthl}
\newcommand{\pthrh}[0]{pthr}
\newcommand{\pointc}[1]{#1_{\pt}} % point constructor
\newcommand{\pathI}[1]{#1_{\pthI}} % path constructor index
\newcommand{\pathA}[1]{#1_{\pthA}} % path constructor index
\newcommand{\pathlh}[1]{#1_{\pthlh}} % left endpoint of path constructor
\newcommand{\pathrh}[1]{#1_{\pthrh}} % left endpoint of path constructor

\newcommand{\comm}{comm}
\newcommand{\trunc}{trunc}

% Algebras
\newcommand{\semP}[1]{\llbracket #1 \rrbracket} % semantics of polynomials
\newcommand{\prealg}[1]{\category{PreAlg}(#1)} % category of prealgebras
\newcommand{\forget}[1]{F_{#1}} % forgetful functor
\newcommand{\semE}[1]{\llbracket #1 \rrbracket} % semantics of endpoints

% Algebras in setoids
\newcommand{\semPT}[1]{\langle #1 \rangle} % semantics of polynomials in setoids
\newcommand{\semET}[1]{\langle #1 \rangle} % semantics of endpoints in setoids

% Displayed algebras
\newcommand{\polydact}[2]{\overline{#1} \> #2}
\newcommand{\epdact}[2]{\overline{#1} \> Y}

\def\lastname{Please list Your Lastname Here}
\begin{document}
\begin{frontmatter}
  \title{An Example Paper} \author{My
    Name\thanksref{ALL}\thanksref{myemail}}
  \address{My Department\\ My University\\
    My City, My Country} \author{My Co-author\thanksref{coemail}}
  \address{My Co-author's Department\\My Co-author's University\\
    My Co-author's City, My Co-author's Country} \thanks[ALL]{Thanks
    to everyone who should be thanked} \thanks[myemail]{Email:
    \href{mailto:myuserid@mydept.myinst.myedu} {\texttt{\normalshape
        myuserid@mydept.myinst.myedu}}} \thanks[coemail]{Email:
    \href{mailto:couserid@codept.coinst.coedu} {\texttt{\normalshape
        couserid@codept.coinst.coedu}}}
\begin{abstract} 
  This is a short example to show the basics of using the ENTCS style
  macro files.  Ample examples of how files should look may be found
  among the published volumes of the series at the ENTCS Home Page
  \texttt{http://www.elsevier.com/locate/entcs}.
\end{abstract}
\begin{keyword}
  Please list keywords from your paper here, separated by commas.
\end{keyword}
\end{frontmatter}
\section{Introduction}
\label{intro}

\subsection*{Preliminaries}

\section{Signature of HITs}
Before we study the construction of set truncated HITs, we must first give a precise definition of those.
We do this by internalizing a scheme, so, more concretely, we define a type of signatures.
In this section, we give the definition of signatures, and then in the upcoming sections, we define the notion of HITs for a signature.

Higher inductive types are freely generated by both point and path constructors.
The data of the point constructor is a \emph{polynomial functor}.
These are described by the following type.

\begin{definition}
We define the type $\mathcal{\poly}$ of \emph{polynomial functors} inductively by the following rules
\begin{center}
\begin{bprooftree}
\AxiomC{$X : \hset$}
\UnaryInfC{$\C \> X : \poly$}
\end{bprooftree}
\begin{bprooftree}
\AxiomC{$\I : \poly$}
\end{bprooftree}
\begin{bprooftree}
\AxiomC{$P : \poly$}
\AxiomC{$Q : \poly$}
\BinaryInfC{$\sumP{P}{Q} : \poly$}
\end{bprooftree}
\begin{bprooftree}
\AxiomC{$P : \poly$}
\AxiomC{$Q : \poly$}
\BinaryInfC{$\prodP{P}{Q} : \poly$}
\end{bprooftree}
\end{center}
\end{definition}

Next we describe the possible path constructors.
Note that these depend on the point constructor.
Furthermore, the path constructors are universally quantified equations of which the arguments depend polynomially on the type being defined.
Again we use an inductive definition to describe all possibilities.

\begin{definition}
Given polynomials, $A, P, Q : \poly$, we define the type $\ep{A}{P}{Q}$ of \emph{endpoints} inductively by the following rules
\begin{center}
\begin{bprooftree}
\AxiomC{$P : \poly$}
\UnaryInfC{$\id{A} : \ep{A}{P}{P}$}
\end{bprooftree}
\begin{bprooftree}
\AxiomC{$P, Q, R : \poly$}
\AxiomC{$e_1 : \ep{A}{P}{Q}$}
\AxiomC{$e_2 : \ep{A}{Q}{R}$}
\TrinaryInfC{$\comp{e_1}{e_2} : \ep{A}{P}{R}$}
\end{bprooftree}
\end{center}
\begin{center}
\begin{bprooftree}
\AxiomC{$P, Q : \poly$}
\UnaryInfC{$\inle : \ep{A}{P}{\sumP{P}{Q}}$}
\end{bprooftree}
\begin{bprooftree}
\AxiomC{$P, Q : \poly$}
\UnaryInfC{$\inre : \ep{A}{Q}{\sumP{P}{Q}}$}
\end{bprooftree}
\begin{bprooftree}
\AxiomC{$P, Q : \poly$}
\UnaryInfC{$\prle : \ep{A}{\prodP{P}{Q}}{P}$}
\end{bprooftree}
\begin{bprooftree}
\AxiomC{$P, Q : \poly$}
\UnaryInfC{$\prre : \ep{A}{\prodP{P}{Q}}{Q}$}
\end{bprooftree}
\end{center}
\begin{center}
\begin{bprooftree}
\AxiomC{$\constr : \ep{A}{A}{\I}$}
\end{bprooftree}
\begin{bprooftree}
\AxiomC{$P : \poly$}
\AxiomC{$X : \hset$}
\AxiomC{$x : X$}
\TrinaryInfC{$\Ce \> x : \ep{A}{P}{\C \> X}$}
\end{bprooftree}
\begin{bprooftree}
\AxiomC{$P, Q, R: \poly$}
\AxiomC{$e_1 : \ep{A}{P}{Q}$}
\AxiomC{$e_2 : \ep{A}{P}{R}$}
\TrinaryInfC{$\pair{e_1}{e_2} : \ep{A}{P}{\prodP{Q}{R}}$}
\end{bprooftree}
\end{center}
\end{definition}

Next we put it all together.
Note that we index the path constructors by a type meaning that we could possibly have infinitely many path constructors.

\begin{definition}
A \emph{HIT signature} $\Sigma$ consists of
\begin{enumerate}
	\item A polynomial $\pointc{\Sigma} : \poly$;
	\item A type $\pathI{\Sigma}$;
	\item A family $\pathA{\Sigma} : \pathI{\Sigma} \rightarrow \poly$;
	\item For each $j : \pathI{\Sigma}$, endpoints $\pathlh{\Sigma} \> j, \pathrh{\Sigma} \> j : \ep{P}{Q \> j}{I}$.
\end{enumerate}
\end{definition}

To illustrate the possibilities of this definition, we consider two examples.
The first one is a set with a commutative operation and the second one is the propositional truncation.

\begin{example}
Define a signature $\comm$ such that
\begin{itemize}
	\item $\pointc{\comm} = \prodP{\I}{\I}$;
	\item $\pathI{\comm} = \1$;
	\item $\pathA{\comm} = \prodP{\I}{\I}$;
	\item $\pathlh{\comm} \> j = \comp{\constr}{\id{\prodP{\I}{\I}}}$;
	\item $\pathrh{\comm} \> j = \comp{\constr}{\pair{\comp{\prle}{\id{\prodP{\I}{\I}}}}{\comp{\prre}{\id{\prodP{\I}{\I}}}}}$;
\end{itemize}
\end{example}

Similarly, one can define signatures for groups or finite sets.
Since signatures can depend on types, we can define the propositional truncation.

\begin{example}
Let $A$ be a set.
Define a signature $\trunc$ such that
\begin{itemize}
	\item $\pointc{\trunc} = \C \> A$;
	\item $\pathI{\trunc} = \1$;
	\item $\pathA{\trunc} = \prodP{\I}{\I}$;
	\item $\pathlh{\trunc} \> j = \comp{\prle}{\id{\prodP{\I}{\I}}}$;
	\item $\pathrh{\trunc} \> j = \comp{\prre}{\id{\prodP{\I}{\I}}}$;
\end{itemize}
\end{example}

\section{Algebras}
Now we know what signatures are, the next goal is to define HITs for signatures.
One possibility for that, is to immediately give the introduction, elimination, and computation rules.
However, we refrain to do so.
Instead we first define a category of algebras and then we say when an algebra is a HIT.

The goal of this section is to define algebras on signatures.
Since ultimately, we want to construct HITs as quotients, we define algebras both in sets and setoids.
The algebras in sets are those types which have the correct introduction rules, while we use those in setoids for the quotient.

Let us start by interpreting polynomials as functors.

\begin{definition}
For each $P : \poly$, we define a functor $\semP{P} : \functor{\hset}{\hset}$. On the objects, they act as follows
\begin{itemize}
	\item $\semP{\C \> X} \> Z = X$;
	\item $\semP{\I} \> Z = Z$;
	\item $\semP{\sumP{P}{Q}} \> Z = \semP{P} \> Z + \semP{Q} \> Z$;
	\item $\semP{\sumP{P}{Q}} \> Z = \semP{P} \> Z \times \semP{Q} \> Z$.
\end{itemize}
\end{definition}

Now we define the category $\prealg{P}$ to be the category of $\semP{P}$-algebras.
Its objects are maps $\semP{P} \> X \rightarrow X$ and, for clarity, we call these \emph{prealgebras}.
Furthermore, note that we have a functor $\forget{P} : \functor{\semP{P}}{\hset}$.

Endpoints are interpreted as natural transformations.
Before we give the precise interpretation, we first need that for every two functors $F, G$ we have transformations
\[
\inlt : \nattrans{F}{\sumf{F}{G}}
\quad \quad
\inrt : \nattrans{G}{\sumf{F}{G}}
\quad \quad
\prlt : \nattrans{\prodf{F}{G}}{F}
\quad \quad
\prrt : \nattrans{\prodf{F}{G}}{G}
\]
In addition, given $\eta_1 : \nattrans{\compf{F}{G_1}}{\compf{F}{G_2}}$ and $\eta_2 : \nattrans{\compf{F}{G_1}}{\compf{F}{G_3}}$, we get $\pairt{\eta_1}{\eta_2} : \nattrans{\compf{F}{G_1}}{\compf{F}{\prodf{G_2}{G_3}}}$.
With these operations in place, we define

\begin{definition}
For each endpoint $e : \ep{A}{P}{Q}$, we define a natural transformation $\semE{e} : \nattrans{\compf{\forget{A}}{\semP{P}}} {\compf{\forget{A}}{\semP{Q}}}$
\begin{itemize}
	\item $\semE{\id{P}} = \idt{\compf{\forget{A}}{\semP{P}}}$;
	\item $\semE{\comp{e_1}{e_2}} = \compt{\semE{e_1}}{\semE{e_2}}$;
	\item $\semE{\inle} = \lwhisker{\forget{A}}{\inlt}$;
	\item $\semE{\inre} = \lwhisker{\forget{A}}{\inrt}$;
	\item $\semE{\prle} = \lwhisker{\forget{A}}{\prlt}$;
	\item $\semE{\prre} = \lwhisker{\forget{A}}{\prrt}$;
	\item $\semE{\pair{e_1}{e_2}} = \pairt{\semE{e_1}}{\semE{e_2}}$;
	\item $\semE{\Ce \> t} = $;
	\item $\semE{\constr} = $.
\end{itemize}
\end{definition}

Now we have everything in place to define algebras on $\Sigma$.
An algebra on $\Sigma$ need to have an operation which satisfies certain equations.
For the operation, we use $\prealg{\pointc{\Sigma}}$.
Since the carrier of each prealgebra is a set, the equations form a proposition.
Hence, we define the category of \emph{$\Sigma$-algebras} as a full subcategory of $\prealg{\pointc{\Sigma}}$.

\begin{definition}
Let $\Sigma$ be a HIT signature.
Then we define the category of \emph{$\Sigma$-algebras} as the full subcategory of $\prealg{\pointc{\Sigma}}$ such that each object satisfies
\[
\deprod{j}{\pathI{\Sigma}}{\deprod{x}{\pathA{\Sigma} \> j}{\semE{\pathlh{\Sigma} \> j} \> x = \semE{\pathrh{\Sigma} \> j} \> x}}.
\]
\end{definition}

In univalent foundations, the 'right' notion of a category is a ``\emph{univalent category}''.
In such categories, the notion of equality corresponds with isomorphism.
More concretely, we have a map $\idtoiso$ which sends equalities $X = Y$ to isomorphisms from $X$ to $Y$, and a category is univalent whenever this map is a weak equivalence.
By the univalence axiom, the category of sets is univalent.

The category of $\Sigma$-algebras also is univalent, which means that isomorphism of algebras corresponds to equality of algebras.
This follows from the factor that the category of algebras on a functor is univalent and that univalence is preserved under taking full subcategories.

\begin{proposition}
The category of $\Sigma$-algebras is univalent.
\end{proposition}

Before we look at algebras in setoids in more detail, we recall the examples in the previous section and look what algebras on those signatures are.

\begin{example}
Recall the signature $\comm$.
An algebra of $\comm$ consists of a set $X$ and a map $f : X \times X \rightarrow X$ such that for all $(x , y) : X \times X$, we have $f(x, y) = f(y, x)$.
\end{example}

\begin{example}
Recall the signature $\trunc \> A$.
An algebra of $\trunc$ consists of a set $X$ and a map $f : A \rightarrow X$ such that for all $(x , y) : X \times X$, we have $x = y$.
In particular, this means $X$ is a mere proposition.
\end{example}

Recall that the core idea is that we define the initial algebra in set by taking the quotient of the initial algebra in setoids.
For that reason, we also need to define algebras in setoids and for these, we take a slightly different approach.
Instead of defining the interpretation of the polynomial $P$ inductively making use of binary sums and products in $\setoids$, we define an equivalence relation on $\semP{P}$, and show this action is functorial.

\begin{definition}
Let $R$ be an equivalence relation on a set $X$ and let $P : \poly$ be a polynomial.
By induction, we define an equivalence relation $P \> R$ on $\semP{P} \> X$.
\begin{itemize}
	\item
	\item
	\item
	\item
\end{itemize}
Now we define a functor $\semPT{P} : \functor{\setoids}{\setoids}$ for which $\semPT{P}(X,R) = (\semP{P} \> X, P \> R)$.
\end{definition}

Again this gives rise to a category $\prealg{P}$

\begin{definition}
Let $e : \ep{A}{P}{Q}$ be an endpoint and let $X$ be a setoid prealgebra on $A$.
Then we define $\semE{e}$ to be the setoid morphism from $\semE{P}(X)$ to $\semE{Q}(X)$ whose carrier is $\semP{e}$.
\end{definition}

\begin{definition}
Let $\Sigma$ be a HIT signature.
Then we define the category of \emph{$\Sigma$-setoid-algebras} as the full subcategory of $\prealg{}$ with respect to
\end{definition}

\begin{proposition}
The category of $\Sigma$-setoid-algebras is unvialent.
\end{proposition}

\section{Initial Algebra Semantics}

\subsection{The Induction Principle}

\begin{definition}
Given are a polynomial $P$ and a family $Y : X \rightarrow \hset$.
We define a family $\polydact{P}{Y} : \semP{P} \> X \rightarrow \hset$ by induction
\begin{itemize}
	\item $\polydact{\C \> X}{Y} \> x = X$;
	\item $\polydact{\I}{Y} \> x = Y \> x$;
	\item $\polydact{\sumP{P}{Q}}{Y} \> (\inl \> x) = \polydact{P}{Y} \> x$;
	\item $\polydact{\sumP{P}{Q}}{Y} \> (\inr \> x) = \polydact{Q}{Y} \> x$;
	\item $\polydact{\prodP{P}{Q}}{Y} \> (\inl \> x) = \polydact{P}{Y} \> (\prr \> x) \times \polydact{Q}{Y} \> (\prl \> x)$.
\end{itemize}
\end{definition}

\begin{definition}
Let $A$ be a polynomial, $(X , f)$ be a prealgebra on $A$, and let $e : \ep{A}{P}{Q}$.
Suppose, we also have a family $Y$ on $X$ and an
\[
c : \deprod{z}{\semP{A}{X}}{\polydact{P}{Y} \> z \rightarrow Y \> (f \> z)}.
\]
We define, by induction on e, a map
\[
\epdact{e}{Y} : \deprod{z}{\semP{P}{X}}{\polydact{P}{Y} \> z \rightarrow \polydact{Q}{Y} \> (\semE{e} \> z)}.
\]
\end{definition}

\begin{definition}
Let $\Sigma$ be a HIT signature and let $X$ be an algebra on $\sigma$.
Then a displayed algebra over $X$ consists of
\begin{itemize}
	\item A type family $Y : X \rightarrow \hset$;
	\item An operation 
	\[
	c : \deprod{z}{\semP{A}{X}}{\polydact{P}{Y} \> z \rightarrow Y \> (f \> z)};
	\]
	\item For each ..., a path
	\[
	\]
\end{itemize} 
\end{definition}

\begin{definition}
Let $Y$ be a displayed algebra over $X$.
Then a displayed algebra map to $Y$ consists of $f : \deprod{x}{X}{Y \> x}$ such that for each $x : ...$, we have
\[
...
\]
\end{definition}

\begin{definition}
Let $\Sigma$ be a HIT signature.
A HIT on $\Sigma$ consists of an algebra $H$ such that for each displayed algebra $Y$ on $H$, we have a displayed algebra map to $Y$.
\end{definition}

\subsection{Obtaining Induction from Initiality}
Now suppose that we have some displayed algebra.
Our goal is to construct an algebra whose objects are dependent pairs.

Give the carrier and operation in plain text

\begin{lemma}
pr1 endpoint
\end{lemma}

\begin{lemma}
pr2 endpoint
\end{lemma}

All in all, we get the following construction.

\begin{theorem}
Total algebra and projection
\end{theorem}

\begin{proposition}
Map to total algebra which projects into identity is section.
\end{proposition}

Hence, to construct a HIT, it suffices to construct an initial object in the category of algebras.

\begin{corollary}
	
\end{corollary}

\section{Constructing the Initial Algebra}

\subsection{Adjunction between Algebras}
To construct the adjunction between the category of algebras, we use two lemmata

\begin{lemma}
Given are categories $\mathcal{C}$ and $\mathcal{D}$, functors $A_{\mathcal{C}} : \functor{\mathcal{C}}{\mathcal{C}}$, $A_{\mathcal{D}} : \functor{\mathcal{D}}{\mathcal{D}}$, and $F : \functor{\mathcal{C}}{\mathcal{D}}$.
Then we get a functor 
\end{lemma}

\begin{lemma}
Given are categories $\mathcal{C}$ and $\mathcal{D}$, functors $A_{\mathcal{C}} : \functor{\mathcal{C}}{\mathcal{C}}$, $A_{\mathcal{D}} : \functor{\mathcal{D}}{\mathcal{D}}$, and an adjunction ....
Suppose, the following diagrams commute

Then we get an adjunction
\end{lemma}

\begin{lemma}
Adjunction between full subcategories
\end{lemma}

Now we apply these to get an adjunction between set HIT-algebras and setoid HIT-algebras.
For this, we do the following steps.

\begin{lemma}
The path setoid commutes with sums and products.

The quotient commutes with sums and products.
\end{lemma}

\begin{lemma}
We have a natural isomorphism (path setoid/quotients commutes with polynomials).
\end{lemma}

Now we can lift the quotient and path setoid.
We can verify the conditions and we get

\begin{lemma}
There is an adjunction between prealgebras ...
\end{lemma}

To show this adjunction lifts to an adjunction between algebras, we need to show that they map algebras to algebras.
We only show this for the path setoid

\begin{lemma}
path setoid endpoint
\end{lemma}

\begin{lemma}
Path setoid factors through algebras
\end{lemma}

All in all, we get

\begin{proposition}
We have an adjunction
\end{proposition}

\subsection{Initial Setoid Algebra}

Hence, we conclude

\begin{theorem}
HITs exist
\end{theorem}

\section{Consequences}

\subsection{Recursion}

\subsection{Uniqueness of HITs}

\begin{proposition}
Induction implies initiality
\end{proposition}

\begin{corollary}
HITs are initial objects
\end{corollary}

\subsection{Path Spaces of HITs}

\section{Conclusion}

\begin{thebibliography}{10}\label{bibliography}
\bibitem{cy} Civin, P., and B. Yood, \emph{Involutions on Banach
    algebras}, Pacific J. Math. \textbf{9} (1959), 415--436.
  
\bibitem{cp} Clifford, A. H., and G. B. Preston, ``The Algebraic
  Theory of Semigroups,'' Math. Surveys \textbf{7}, Amer. Math. Soc.,
  Providence, R.I., 1961.
  
\bibitem{f} Freyd, Peter, Peter O'Hearn, John Power, Robert Tennent
  and Makoto Takeyama, \emph{Bireflectivity}, Electronic Notes in
  Theoretical Computer Science {\bf 1} (1995), URL:
  \href{https://www.sciencedirect.com/journal/electronic-notes-in-theoretical-computer-science/vol/1/suppl/C}
  {\texttt{http://www.elsevier.com/locate/entcs/volume1.html}}.
  
\bibitem{em2} Easdown, D., and W. D. Munn, \emph{Trace functions on
    inverse semigroup algebras}, U. of Glasgow, Dept. of Math.,
  preprint 93/52.

\bibitem{r} Roscoe, A. W., ``The Theory and Practice of Concurrency,''
  Prentice Hall Series in Computer Science, Prentice Hall Publishers,
  London, New York (1198), 565pp. With associated web site\\  
  \href{http://www.comlab.ox.ac.uk/oucl/publications/books/concurrency/}
  {\texttt{http://www.comlab.ox.ac.uk/oucl/publications/books/concurrency/}}.
  
\bibitem{s} Shehadah, A. A., ``Embedding theorems for semigroups with
  involution, `` Ph.D.  thesis, Purdue University, Indiana, 1982.
  
\bibitem{w} Weyl, H., ``The Classical Groups,'' 2nd Ed., Princeton U.
  Press, Princeton, N.J., 1946.

\end{thebibliography}

\end{document}
